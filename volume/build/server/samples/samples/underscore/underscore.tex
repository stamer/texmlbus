\documentclass[a4paper]{article}
\usepackage[a4paper]{geometry}
\usepackage{miscdoc}
%\usepackage[scaled=0.85]{luximono}
\usepackage{underscore}
\begin{document}
\title{The \Package{underscore} package}
\author{Donald Arseneau\thanks{Documentation file assembled by Robin
    Fairbairns}}
\date{13 September 2006, version 1.0}
\maketitle

\section*{Features}

The package alters the command \cs{_} (which normally prints
an underscore character or facsimile) so that the hyphenation of
constituent words is not affected, and hyphenation is permitted after
the underscore.  For example,
\begin{quote}
  \texttt{compound\cs{_}fracture}
\end{quote}
hyphenates as
\begin{quote}
  \texttt{com- pound\cs{_}- frac- ture}
\end{quote}
If you prefer the underscore to break without a hyphen (but still with
the same rules as for explicit hyphen-breaks) then use the
\option[nohyphen] package option.

A simple ``\texttt{_}'' acts just like ``\cs{_}'' in text
mode, but makes a subscript in maths mode, so
\begin{quote}
  \ttfamily activation_energy \$E_a\$
\end{quote}
is printed as
\begin{quote}
  activation_energy $E_a$
\end{quote}

Both forms use an underscore character if the font encoding contains
one (e.g., with \cmdinvoke{usepackage}[T1]{fontenc} or typewriter
fonts in any encoding), but they use a rule if there is no proper
character (just as unmodified \LaTeX{} does).

\section*{Deficiencies}

The skips and penalties ruin any kerning with the underscore character
(when a character is used).  However, there doesn't seem to be much, if
any, such kerning in the EC fonts, and there is never any kerning with
a rule.

You must avoid ``\texttt{_}'' in file names and in cite or ref tags, or you must
use the \Package{babel} package, with its active-character controls,
or you must give the \option[strings] option, which attempts to
redefine several commands (and may not work perfectly).  Even without
the \option[strings] option or \Package{babel}, you can use occasional
underscores like: ``\cmdinvoke{include}{file\string\string_name}''.

\subsection*{Option \protect\option[strings]}

The default operation is quite simple and needs no customization; but
you must avoid using ``\texttt{_}'' in any place where LaTeX uses an argument
as a string of characters for some control function or as a name.
These include the tags for \cs{cite} and \cs{ref}, file names for \cs{input}, 
\cs{include}, and \cs{includegraphics}, environment names, counter names,
and placement parameters (like \option[t]).  The problem with these contexts
is that they are `moving arguments' but LaTeX does not `switch on' the
``\cs{protect} mechanism'' for them.

If you need to use the underscore character in these places, the
package option \option[strings] is provided to redefine commands that
take such a string argument so that protection is applied (with
\cs{protect} made to be \cs{string}).  The list of commands this
provision affects is given in \cs{UnderscoreCommands}, with \cs{do}
before each one; plus several others covering \cs{input},
\cs{includegraphics}, \cs{cite}, \cs{ref}, and their variants.  Not
included are many commands regarding font names, anything with counter
names, environment names, page styles, and versions of \cs{ref} and
\cs{cite} defined by external packages (e.g., \cs{vref} and
\cs{citeyear}).

You can add to the list of supported commands by defining \cs{UnderscoreCommands}
before loading this package; e.g.
\begin{quote}
\begin{verbatim}
\usepackage{chicago}
\newcommand{\UnderscoreCommands}{%   (\cite already done)
  \do\citeNP \do\citeA \do\citeANP \do\citeN \do\shortcite
  \do\shortciteNP \do\shortciteA \do\shortciteANP \do\shortciteN
  \do\citeyear \do\citeyearNP
}
\usepackage[strings]{underscore}
\end{verbatim}
\end{quote}

Not all commands can be supported this way!  Only commands that take a
string argument \emph{first} can be protected.  One optional argument before
the string argument is also permitted, as exemplified by \cs{cite}: both
\cmdinvoke{cite}{tags} and \cmdinvoke{cite}[text]{tags} are allowed.
A command like \cs{@addtoreset} which takes two counter names as
arguments could not be protected by listing it in \cs{UnderscoreCommands}.

\textit{When you use the \option[strings] option, you must load this
  package \textbf{last}} (or nearly last).
There are two reasons for this requirement: 
\begin{enumerate}
\item The redefinitions done for protection must come after other
  packages define their customized versions of those commands.
\item The \option[strings] option requires the ``\texttt{_}'' character to be
  activated immediately in order for the cite and ref tags to be read
  properly from the \texttt{.aux} file as plain strings, and this
  catcode setting might disrupt other packages.
\end{enumerate}

The \Package{babel} package implements a protection mechanism for many
commands, and will be a complete fix for most documents without the
\option[strings] option.  Many add-on packages are compatible with
\Package{babel}, so they will get the strings protection also.
However, there are several commands that are not covered by
\Package{babel}, but can easily be supported by \option[strings] and 
\cs{UnderscoreCommands} mechanism.  Beware the potential conflict
using both \option[strings] and \Package{babel} (though none have been
reported, yet); load babel last.

\subsection*{Implementation notes}

The first setting of ``\texttt{_}'' to be an active character is
performed in a local group so as to not interfere with other packages.
The catcode setting is repeated with ``\cs{AtBeginDocument}'' so the
definition is in effect for the text.  However, the catcode setting is
repeated immediately when the \option[strings] option is detected.

The definition of the active ``\texttt{_}'' is essentially:
\begin{quote}
  \cs{ifmmode} \cs{sb} \cs{else} \cs{BreakableUnderscore} \cs{fi}
\end{quote}
where \cs{sb} retains the normal subscript meaning of \texttt{_} and where
\cs{BreakableUnderscore} is essentially \cs{_}.  The rest of the definition
handles the \cs{protect}ion without causing \cs{relax} to be inserted before
the character.

\cs{BreakableUnderscore} uses \cs{nobreak}\cs{hskip}\cs{z@skip} to separate the
underscore from surrounding words, thus allowing \TeX{} to hyphenate them,
but preventing free breaks around the underscore.  Next, it checks the
current font family, and uses the underscore character from tt fonts or
otherwise \cs{textunderscore} (which is a character or rule depending on
the font encoding).  After the underscore, it inserts a discretionary
hyphenation point as \cs{usc@dischyph}, which is usually just \cs{-}
except that it still works in the tabbing environment; if the
\option[nohyphen] option is in effect, the empty discretionary
\cmdinvoke{discretionary}{}{}{} is used instead.  After
that, another piece of non-breaking interword glue is inserted. 

Ordinarily, the comparison \cs{ifx}\cs{f@family}\cs{ttdefault} will fail 
because \cs{ttdefault} is `long' whereas \cs{f@family} is
not\footnote{the package author says ``boooo hisss'' about this\dots},
but \cs{ttdefault} is redefined to be non-long \cs{AtBeginDocument}.

The \cs{_} command is then defined to use \cs{BreakableUnderscore}.

If the \option[strings] option has not been given, that is all!

Under the \option[strings] option, the list of special commands is processed to:
\begin{itemize}
\item retain the original command as \cs{US_}\meta{command} (e.g., \cs{US_ref})
\item redefine the command as \cs{US@prot}\cs{US_command} for ordinary
  commands (\cs{US@prot}\cs{US_ref}) or as
  \cs{US@protopt}\cs{US_command} when an optional argument is possible
  (e.g., \cs{US@protopt}\cs{US_bibitem}). 
\item self-protecting commands (e.g., \cs{cite}) retain their
  self-protection.
\end{itemize}
Diagnosing the state of the pre-existing command is done by painful
contortions involving \cs{meaning}.

\cs{US@prot} and \cs{US@protopt} read the argument, process it with
\cs{protect} enabled, then invoke the saved \cs{US_command}.

\end{document}

%%% Local Variables: 
%%% mode: latex
%%% TeX-master: t
%%% End: 
